

% --------------------------------------------------------------
%      = Header =
% --------------------------------------------------------------

\documentclass{amsart} % AMS article style

\usepackage{amsfonts} % for \mathbb, etc

\usepackage{hyperref}     % for linking to web pages (hyperlinks)

% --------------------------------------------------------------
%      = Newcommands  =
% --------------------------------------------------------------


% ------------------------------------------
% brackets
% ------------------------------------------
% angled brackets

\newcommand{\<}{\langle }  
\renewcommand{\>}{\rangle }  
% Angle Brackets: Angle brackets in LaTeX are not the same as the
% inequality symbols: the angle bracket characters are '〈' and '〉',
% not '<' and '>'. You get the angle brackets using the following
% commands in math mode: \langle and \rangle. 

% https://web.science.mq.edu.au/~rdale/resources/writingnotes/latexstyle.html

% remark. \> is already defined in LaTeX, but I never use it:
% \> ( or \: ) inserts a .22em space in text mode, or \medmuskip
% there's an equivalent \medspace


% ------------------------------------------
% letters
% ------------------------------------------

\newcommand{\ZZ}{\mathbb{Z}}
\newcommand{\QQ}{\mathbb{Q}}
\newcommand{\RR}{\mathbb{R}}
\newcommand{\CC}{\mathbb{C}}

\newcommand{\FF}{\mathbb{F}}  % finite field

\newcommand{\Qbar}{\overline{\QQ}}     % Qbar -- algebraic closure of Q

\newcommand{\HH}{\mathbb{H}}    % hypercohomology groups

\newcommand{\OO}{\mathcal{O}} % Structure sheaf

% rare

\newcommand{\adelicA}{\mathbb{A}}  % actually, same as the affine space

% ------------------------------------------
% =  new arrows = 
% ------------------------------------------

% an arrow with something above it
\newcommand{\arrow}[1]{\stackrel{#1}{\to}}  
           % example: A \arrow{\alpha} B
%
% equality sign '=', with something above it,
\newcommand{\equal}[1]{\stackrel{#1}{=}}  
           % example: a^2 + b^2 \equalEx{[1]}  c^2

\newcommand{\equalEx}{\equal}

\newcommand{\Isom}{\stackrel{\sim}{\to}}    % arrow -> with tilde
                                            % above


% in older texts I called it \isom, with small 'i'

\newcommand{\IsomRl}{\stackrel{\sim}{\leftarrow}}       %  arrow <- with tilde
\newcommand{\isom}{\simeq}                      

\newcommand{\rationalMap}{\dashrightarrow}     % dashed arrow - ->

\newcommand{\equalDef}{\stackrel{\rm def}{=}}       %  equal by definition

% ``arrows``, logical: 

\newcommand{\equivalent}{\Longleftrightarrow}  % ⟺
\renewcommand{\equiv}{\equivalent}  % used to be number-theoretic ≡

\newcommand{\shortEquiv}{\Leftrightarrow}  % shorter form

                                               

% -- binary operations, like A + B
\newcommand{\intersect}{\cap}                       % intersection sign
\newcommand{\intersection}{\cap}                    % intersection sign

% ----------------------------------------------------------------------
%     == structures with mathop ==
% ----------------------------------------------------------------------

% ---------- to redo these:

% -- binary operations, like A + B

\newcommand{\tensor}{\mathop{\otimes}}
\newcommand{\directsum}{\mathop{\oplus}}         % direct-sum is not allowed
\newcommand{\union}{\mathop{\cup}}
\newcommand{\fibered}{\mathop{\times}\limits}  % fibered product 

\newcommand{\disjointUnion}{\mathop{\sqcup}}

% math operations, like \dim (which is built-in):  


\newcommand{\codim}{\mathop{\rm codim}\nolimits}
\newcommand{\rk}{\mathop{\rm rk}}
\newcommand{\im}{\mathop{\rm Im}}    % \Im is something different in latex 
\newcommand{\coker}{\mathop{\rm Coker}}

% \newcommand{\tr}{\mathop{\rm tr}}


\newcommand{\Tors}{\mathop{\rm Tors}}
\newcommand{\Supp}{\mathop{\rm Supp}}

\newcommand{\Ext}{\mathop{\rm Ext}\nolimits}
\newcommand{\Hom}{\mathop{\rm Hom}\nolimits}
\newcommand{\End}{\mathop{\rm End}\nolimits}

\newcommand{\Spec}{\mathop{\rm Spec}}
\providecommand{\Proj}{\mathop{\rm Proj}}

       % projective space;
       % also: projectivization of a vector bundle;
\newcommand{\PP}{\mathop{\mathbb{P}}\nolimits}  
\newcommand{\pp}{\PP}  

       % affine space; originally in Latex it is Angstrom...
\renewcommand{\AA}{\mathbb{A}}  

\newcommand{\Bl}{\mathop{\rm Bl}\nolimits}  % blowup

\newcommand{\Sing}{\mathop{\rm Sing \;}\nolimits}  % subscheme of singularities
\newcommand{\sing}{\Sing}  

%                  moduli spaces

\newcommand{\Pic}{\mathop{\rm Pic}\nolimits} % Picard group or variety
\newcommand{\pic}{\Pic}
\newcommand{\Hilb}{\mathop{\rm Hilb}\nolimits}
\newcommand{\hilb}{\Hilb}


% div(f) for a function f
% div_X(f) on X
\renewcommand{\div}{\mathop{\mathrm{div}}\nolimits}

\newcommand{\Sym}{\mathop{\rm Sym}\nolimits}   % Symmetric power
\newcommand{\sym}{\mathop{\rm sym}\nolimits}    % same, small 's

\newcommand{\FT}{\mathop{\mathrm{FT}}\nolimits} % Fourier transform

\newcommand{\id}{ {\rm id} }  % identity map

% equivalences
\newcommand{\numEquiv}{ \mathop{\sim}\limits_{\rm num} } % numerically equivalent

% symbols
\newcommand{\integral}{\int}      
\newcommand{\orthogonal}{^{\perp}}      % orthogonal complement 
\newcommand{\ort}{\orthogonal}      
\newcommand{\dual}{^{\vee}}
\newcommand{\dbar}{\bar{\partial}}       % d bar operator
\newcommand{\pmNew}{\raisebox{.2ex}{$\scriptstyle\pm$}}  % better plus-minus sign, from Stack Overflow
\newcommand{\pmCentered}{\pmNew}




% complexes
\newcommand{\cOmega}{\Omega^{\cdot}}        % (de Rham) complex Omega (with dot)
                           % tried {}^{\cdot}, but then the dot is too small

% sheaves
\newcommand{\sheafHom}{\mathcal{H} \mathit{om}}   % sheaf Hom
\newcommand{\sheafExt}{\mathcal{E} \mathit{xt}}   % sheaf Ext 
\newcommand{\sheafH}{\mathcal{H}}                 % sheaf H


%  -- foreign names (incl. French names)

\newcommand{\Chech}{\v{C}ech}    
% another web page suggests: \v{C}ech
\newcommand{\ChechIndex}{_{\check{C}}}  

\newcommand{\Neron}{N\'eron}
\newcommand{\Bezout}{B\'ezout}

% --------------------------------------------------------------
% -- document --

\begin{document}

%  -- title --

\title{Geometric introduction to trigonal curves of genus five}
\author{Maxim Leyenson}
\date{}   % empty date 

\maketitle
%  -- end of: title --

%-------------------toc ---------------------
% \tableofcontents
%----------------------------------------


% ----------------------------------------------------------------------
% =  TeX file, body =
% ----------------------------------------------------------------------

\section{}%
\label{sec:section name}

%1.
Let $C$ be a non-hyperelliptic trigonal curve of genus five (assume for a moment that it
exists; we will construct one in a few moments). Then it has a plane
model $C'$ which is a quintic with a double point:

Indeed, we can write the trigonal structure as $g_3^1 = |p + q + r|$, where the points p, q and r on $C$ are distinct.
(In other words, a trigonal structure is a map \[
  \pi: C \to \PP^1,
\], and for some point $a \in \PP^1$ the preimage of the point $a$
consists of three separate points, $\pi^{-1}(a) = p + q + r$.)

Then, by the geometric version of the Riemann-Roch theorem, the points p, q and r 
all belong to the same line $l$ in the canonical embedding of $C$. 

When we vary (move) the divisor $D = p + q + r$ in the linear system
$g_3^1$, the linear span $\<D\>$, which is a line,  is changing, too; forming a ruled surface
in $\PP^4$ containing the curve $C$. More on this later; this surface
containing $C$ is important.
  
Now consider the linear system of hyperplanes $|H|$ in $\PP^4$ and the linear subsystem
$|H - l|$ of hyperplanes containing the fixed line $l$ for some
divisor $D = p + q + r $ in the linear system $g_3^1$. This linear
system of hyperplanes cuts out a linear system  
$|K_C - p - q - r| $ of dimension $4 - ( \dim l + 1) = 2$ and degree 8 - 3 = 5 on $C$:

\[
    K_C - g_3^1 = g_5^2
\]
This $g_5^2$ gives a plane model $C'$ of the curve $C$ which is a curve of degree 5;
by construction, it is just the projection of the curve $C$ with the center $l$.

Since $p_a(C') = 4 \cdot 3 / 2 = 6$, we have $ \delta(C') = p_a(C') - p_g(C') = 
6 - 5 = 1$, and thus $C'$ is a singular curve with one singular point,
which can be either a node or one cusp.
(See Serre, "Algebraic groups and class fields" for an excellent introduction to the $\delta$ - invariant of
singular curves.)

% ---------------------------------------------------------------------

\section{}%
\label{sec:}

% 2. 

We can now prove the existence of such a curve $C$. Consider a plane curve $C'$
of degree 5 with one node $p$. (We omit the case of a cusp.) It is clear that such a curve exists.

The geometric genus of $C'$ is $4 \cdot 3 / 2 - 1 = 5$. Let $C$ be the smooth model of $C'$.
It can be constructed by blowing up the point $p$ on $\PP^2$, 
% $\Bl_p(\PP^2)$, 
and taking the proper preimage of the singular curve $C'$.

It is immediately clear that the curve $C$ is trigonal: the linear system of lines
on the plane through the point $p$, $|l - p|$, gives a pencil of (effective) divisors
of degree 3 on $C$ (and gives a projection of $C$ to a projective line
which is a degree 3 map).

% ---------------------------------------------------------------------- 
\section{}%
\label{sec:}

% 3. 
The linear system of conics through $p$ on the plane, $|2 l - p|$,
cuts out a linear system of degree 10 on $C'$. The base locus of this linear 
system is $2p$. By the adjunction formula for nodal curves, (cf. 
Enrico Arbarello, Maurizio Cornalba, Phillip Griffiths, Joseph Harris,
Geometry of Algebraic Curves: Volume I, Appendix A), after removing the base point $2p$, 
we get the {\it canonical } linear system on $C$. (For any plane nodal curve $C'$,
the canonical class of its smooth model is $ (d-3) l - \{ $ 0-cycle of nodes on  $C' \} $.)

Note that the linear system $|2l - p|$ embeds the blown-up plane $\Bl_p(\PP^2)$ as a ruled surface in $\PP^4$.

This gives an embedding

\[
   i(C) \; \subset i(\Bl(\PP^2)) \; \subset \PP^4 = |K_C|^*
\]

% where $i$ is 
given by the linear system $|2l - p|$.

Note that any line $l_0$ on $\PP^2$ via p becomes a line $L_0$ under the embedding
$i$. (It follows, or example, from the computation $C \cdot l_0 = p + p'$ for
any conic $C$ in $|2l - p|$.

It follows that the image $ S = \Bl(\PP^2) $ in $\PP^4$  is a ruled surface, with every line 
of the ruling giving a trisecant to the $ i(C) $. Thus, $ S $ is the spanned by the trisecant lines
to $ C $.

% ---------------------------------------------------------------------- 

%5. 
\section{}%
\label{sec:section name}

It is easy to see that $ S = \Bl(\PP^2) $ is the intersection of quadrics through $ S $ (and even
through $ C $, as we will see in a moment):

Note that $ i(C) $ is not an intersection of quadrics: for a trisecant line $ L $,
$ L$ intersects any quadric  $ Q $ through $C$ at three points, $  i(p1),.., i(p3) $ ,
and this is contained in $ Q $ . Thus, $ S $ is the intersection of quadrics through
$ C $ .

\section{}%
\label{sec:}


% ---------------------------------------------------------------------- 

%6. 

In fact, the surface $S$ is clearly one and the same for all the
curves $C'$ with a node
at the same point $p$  . A computation with Riemann-Roch theorem gives that every $C$ is cut out in $ S $ 
by its own pencil of cubic hypersurfaces in $\PP^4$. The attached program computes this pencil.


\vspace{10mm} %10 mm vertical space
% ---------------------------------------------------------------------- 
Author: Maxim Leyenson, email address: $\<$ my last name $ \> $ at gmail dot com 


\vspace{10mm} %10 mm vertical space
% ---------------------------------------------------------------------- 

\end{document}

